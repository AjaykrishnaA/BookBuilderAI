\documentclass{book}
\usepackage{graphicx}
\usepackage{listings}
\usepackage{xcolor}
\usepackage{hyperref}
\usepackage{geometry}
\geometry{a4paper, margin=1in}

\title{Sustainable and Readable Programming: Best Practices}
\author{}
\date{}

\definecolor{codegreen}{rgb}{0,0.6,0}
\definecolor{codegray}{rgb}{0.5,0.5,0.5}
\definecolor{codepurple}{rgb}{0.58,0,0.82}
\definecolor{backcolour}{rgb}{0.95,0.95,0.92}

\lstdefinestyle{codestyle}{
    backgroundcolor=\color{backcolour},
    commentstyle=\color{codegreen},
    keywordstyle=\color{magenta},
    numberstyle=\tiny\color{codegray},
    stringstyle=\color{codepurple},
    basicstyle=\footnotesize,
    breakatwhitespace=false,         
    breaklines=true,                 
    captionpos=b,                    
    keepspaces=true,                 
    numbers=left,                    
    numbersep=5pt,                  
    showspaces=false,                
    showstringspaces=false,
    showtabs=false,                  
    tabsize=4
}

\lstset{style=codestyle}

\begin{document}

\maketitle

\tableofcontents

\chapter{Introduction: The Importance of Sustainable and Readable Code}

\section{What is Sustainable Code?}

Sustainable code is code that can be maintained, updated, and extended over a long period without becoming overly complex or fragile. It considers the environmental, economic, and social impact of software development.

\section{What is Readable Code?}

Readable code is code that is easy to understand, follow, and modify. It prioritizes clarity and maintainability over cleverness or conciseness.

\section{Why are Sustainability and Readability Important?}
\begin{itemize}
    \item Reduced Technical Debt
    \item Lower Maintenance Costs
    \item Increased Collaboration
    \item Environmental Considerations (Energy Efficiency)
\end{itemize}

\chapter{Coding Style and Conventions}

\section{Naming Conventions}

Choose descriptive and meaningful names for variables, functions, and classes.  Avoid single-letter variable names except for loop counters.

\section{Comments and Documentation}

Write clear and concise comments to explain the purpose and functionality of your code.  Use documentation generators (e.g., Doxygen, Sphinx) for larger projects.

\section{Code Formatting}

Use consistent indentation, spacing, and line breaks to improve readability.  Follow a consistent coding style guide (e.g., PEP 8 for Python, Google Style Guide for C++).

\begin{lstlisting}[language=Python, caption=Example of well-formatted code, basicstyle=\footnotesize]
def calculate_area(length, width):
    """Calculates the area of a rectangle.

    Args:
        length: The length of the rectangle.
        width: The width of the rectangle.

    Returns:
        The area of the rectangle.
    """
    area = length * width
    return area


# Example usage
rect_length = 10
rect_width = 5
rectangle_area = calculate_area(rect_length, rect_width)
print(f"The area of the rectangle is: {rectangle_area}")
\end{lstlisting}

\chapter{Design Principles for Sustainability}

\section{SOLID Principles}
\begin{itemize}
    \item Single Responsibility Principle
    \item Open/Closed Principle
    \item Liskov Substitution Principle
    \item Interface Segregation Principle
    \item Dependency Inversion Principle
\end{itemize}

\section{DRY (Don't Repeat Yourself)}

Avoid code duplication by extracting common logic into reusable functions or classes.

\section{KISS (Keep It Simple, Stupid)}

Favor simple and straightforward solutions over complex and convoluted ones.

\chapter{Testing and Quality Assurance}

\section{Unit Testing}

Write unit tests to verify the correctness of individual functions or classes.

\section{Integration Testing}

Test the interactions between different components of your system.

\section{Code Reviews}

Conduct regular code reviews to identify potential issues and improve code quality.

\chapter{Energy Efficiency in Software}

\section{Algorithm Optimization}

Choose efficient algorithms and data structures to minimize resource consumption.

\section{Reducing Memory Footprint}

Optimize memory usage to reduce energy consumption and improve performance.

\section{Hardware Considerations}

Be mindful of the energy efficiency of the hardware your software runs on.


\chapter{Tools and Technologies}
\section{Static Analysis Tools}
\section{Linters}
\section{Performance Profilers}

\chapter{Case Studies}
\section{Open Source Projects}
\section{Industry Examples}

\chapter{Conclusion}

\section{The Future of Sustainable Programming}

\end{document}